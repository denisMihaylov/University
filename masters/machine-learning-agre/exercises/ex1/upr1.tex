\documentclass{article}
\usepackage[utf8]{inputenc}
\usepackage[russian]{babel}
\usepackage{graphicx}
\usepackage{csquotes}
\usepackage{pgf,tikz}
\usepackage{mathrsfs}
\usetikzlibrary{arrows}
\pagestyle{empty}
\definecolor{wwffqq}{rgb}{0.4,1,0}
\definecolor{rvwvcq}{rgb}{0.08235294117647059,0.396078431372549,0.7529411764705882}
\definecolor{ffqqqq}{rgb}{1,0,0}
\definecolor{cqcqcq}{rgb}{0.7529411764705882,0.7529411764705882,0.7529411764705882}
\graphicspath{ {/home/denis/Desktop/machine-learning-agre/exercises/ex1/images/} }

\begin{document}
\title{Упражнение 1}
\author{Денис Симеонов Михайлов \\ ФН: 25788}
\date{23 октомври 2017г.}
\maketitle
\paragraph{Задача}
Представете си, че пространството на примери се състои от точки $<x, y>$ с
целочислени координати, а пространството на хипотези $H$ се състои от
правоъгълници. По-точно, хипотезите се 
записват във вида
\[a \leq x \leq b, c \leq y \leq d\]
където $a$, $b$, $c$ и $d$ са цели числа.

\begin{enumerate}
\item Разгледайте пространството на версиите по отношение на положителните (+) и отрицателните (о) обучаващи примери, показани по-долу. Коя е $S$ границата на пространството на версиите в този случай? Напишете хипотезите и ги нанесете на рисунката.\\
\definecolor{rvwvcq}{rgb}{0.08235294117647059,0.396078431372549,0.7529411764705882}\definecolor{ffqqqq}{rgb}{1,0,0}\definecolor{cqcqcq}{rgb}{0.7529411764705882,0.7529411764705882,0.7529411764705882}\begin{tikzpicture}[line cap=round,line join=round,>=triangle 45,x=1cm,y=1cm]\draw [color=cqcqcq,, xstep=1cm,ystep=1cm] (-2.3423880047878307,-0.4594847187514599) grid (12.648318654781852,8.48412332221215);\draw[->,color=black] (-2.3423880047878307,0) -- (12.648318654781852,0);\foreach \x in {-2,-1,1,2,3,4,5,6,7,8,9,10,11,12}\draw[shift={(\x,0)},color=black] (0pt,2pt) -- (0pt,-2pt) node[below] {\footnotesize $\x$};\draw[->,color=black] (0,-0.4594847187514599) -- (0,8.48412332221215);\foreach \y in {,1,2,3,4,5,6,7,8}\draw[shift={(0,\y)},color=black] (2pt,0pt) -- (-2pt,0pt) node[left] {\footnotesize $\y$};\draw[color=black] (0pt,-10pt) node[right] {\footnotesize $0$};\clip(-2.3423880047878307,-0.4594847187514599) rectangle (12.648318654781852,8.48412332221215);\begin{scriptsize}\draw [color=ffqqqq] (5,1) circle (2.5pt);\draw [color=ffqqqq] (1,3) circle (2.5pt);\draw [color=ffqqqq] (2,6) circle (2.5pt);\draw [color=ffqqqq] (5,8) circle (2.5pt);\draw [color=ffqqqq] (9,4) circle (2.5pt);\draw [color=rvwvcq] (5,3)-- ++(-3pt,0 pt) -- ++(6pt,0 pt) ++(-3pt,-3pt) -- ++(0 pt,6pt);\draw [color=rvwvcq] (4,4)-- ++(-3pt,0 pt) -- ++(6pt,0 pt) ++(-3pt,-3pt) -- ++(0 pt,6pt);\draw [color=rvwvcq] (6,5)-- ++(-3pt,0 pt) -- ++(6pt,0 pt) ++(-3pt,-3pt) -- ++(0 pt,6pt);\end{scriptsize}\end{tikzpicture}

\item 
Коя е $G$ границата на това пространство на версиите? Напишете хипотезите и ги нанесете на рисунката.
\paragraph{Решение}
Ще използваме алгоритъма за елиминиране на кандидати.\\\\
Първо ще дефинираме релацията „по-обща или равна на“ за хипотезите от H. Ще казваме, че $h_1$ от вида:
\[a_1 \leq x \leq b_1, c_1 \leq y \leq d_1\] 
e „по-обща или равна на“ $h_2$ от вида:
\[a_2 \leq x \leq b_2, c_2 \leq y \leq d_2\] 
ако е изпълнено:
\[
	\begin{array}{lr}
		a_1 \leq a_2\\
		b_1 \geq b_2\\
		c_1 \leq c_2\\
		d_1 \geq d_2
	\end{array}
\]
Бележим с $h_1 \geq h_2$\\\\
Построяваме множеството с най-общите хипотези в $H$:
\[G = \{-\infty \leq x \leq \infty, -\infty \leq y \leq \infty\}\]
В $G$ попада единствено хипотезата, което обхваща всички точки в пространството \\\\
Построяваме и множеството $S$ от най-специфичните хипотези в $H$:
\[S = \{\infty \leq x \leq -\infty, \infty \leq y \leq -\infty\}\]
В $S$ попада възможно най-специфичната хипотеза от $H$. Всяка друга хипотеза от $H$ е „по-обща от“ нея.\\\\
Нека първо разгледаме положителните примери. При настъпването на обучаващия пример $<5, 3>$ $G$ не се променя, защото хипотезата в $G$ е съвместима с всички точки в пространството. Хипотезата в $S$ не е съвместима с нито една точка от пространството и нейното най-малко обобщение, което е съвместимa с $<5, 3>$ е 
\[5 \leq x \leq 5, 3 \leq y \leq 3\]
Хипотезата от $G$ е „по-обща или равна на“ нея, защото тя е по-обща от всяка друга хипотеза.
Така множеството $S$ вече има вида:
\[S = \{5 \leq x \leq 5, 3 \leq y \leq 3\}\]
Нека следващият обучителен пример да е $<4, 4>$. $G$ отново не се променя, а хипотезата в $S$ е несъвместима с тази точка. Това означава, че тя трябва да бъде премахната, а на нейно място да се добави най-малкото и обощение, такова че да е съвместимо с $<4, 4>$. Това най-малко обощение е 
\[4 \leq x \leq 5, 3 \leq y \leq 4\]
Това, което направихме, беше да увеличим интервала $[5, 5]$ за $x$ до най-малкия възможен, който включва 4. Това е интервалът $[4, 5]$. Аналогично постъпваме и за $y$. По този начин $S$ придобива вида:
\[S = \{4 \leq x \leq 5, 3 \leq y \leq 4\}\]
След още една аналогична итерация за точката $<6, 5>$, $S$ има вида:
\[S = \{h_s = 4 \leq x \leq 6, 3 \leq y \leq 5\}\]
Нека сега видим какво се случва когато обработваме отрицателните обучителни примери.
Нека първо постъпи примера $<5, 1>$.Хипотезата от $G$ е несъвместима с този пример. Затова трябва да бъде изтрита и на нейно място да се появят нейните най-малки специализации.
Това са хипотезите:
\[
	\begin{array}{lr}
		h_1 = -\infty \leq x \leq 4, -\infty \leq y \leq \infty\\
		h_2 = 6 \leq x \leq \infty, -\infty \leq y \leq \infty\\
		h_3 = -\infty \leq x \leq \infty, 2 \leq y \leq \infty\\
		h_4 = -\infty \leq x \leq \infty, -\infty \leq y \leq 0\\
	\end{array}
\]
От тези четири хипотези само $h_3 \geq h_s$ и затова само тя ще бъде добавена към $G$. По този начин $G$ e придобива вида:
\[G = \{h_g = -\infty \leq x \leq \infty, 2 \leq y \leq \infty\}\]
Нека следващият обучителен пример да е $<1, 3>$. Той също е несъвместим с хипотезата $h_g$. Аналогично на предния обучаващ пример, хипотезата $h_g$ има следните най-малки специализации:
\[
	\begin{array}{lr}
		h_1 = -\infty \leq x \leq 0, 2 \leq y \leq \infty\\
		h_2 = 2 \leq x \leq \infty, 2 \leq y \leq \infty\\
		h_3 = -\infty \leq x \leq \infty, 2 \leq y \leq 2\\
		h_4 = -\infty \leq x \leq \infty, 4 \leq y \leq \infty\\
	\end{array}
\]
От тези само $h_2 \geq h_s$ и затова тя ще бъде добавена към $G$. Вида на $G$ e:
\[G = \{h_g = 2 \leq x \leq \infty, 2 \leq y \leq \infty\}\]
Следващият обучителен пример е $<2, 6>$. $h_g$ отново е несъвместима с него и нейната най-малката специализация на $h_g$, такава че $h_s$ да е по-специфична от нея, е:
\[2 \leq x \leq \infty, 2 \leq y \leq 5\]
По този начин $G$ придобива вида:
\[G = \{h_g = 2 \leq x \leq \infty, 2 \leq y \leq 5\}\]
Следващият пример е $<5, 8>$, но $h_g$ и $h_s$ са съвместими с него затова нищо не се променя.
Последният пример е $<9, 4>$. $h_g$ e несъвместима с него и нейната най-малка специализация е:
\[2 \leq x \leq 8, 2 \leq y \leq 5\]
И така крайният вид на множествата е следния:
\[G = \{h_g = 2 \leq x \leq 8, 2 \leq y \leq 5\}\]
\[S = \{h_s = 4 \leq x \leq 6, 3 \leq y \leq 5\}\]
\begin{tikzpicture}[line cap=round,line join=round,>=triangle 45,x=1cm,y=1cm]\draw [color=cqcqcq,, xstep=1cm,ystep=1cm] (-2.3423880047878307,-0.4594847187514599) grid (12.648318654781852,8.48412332221215);\draw[->,color=black] (-2.3423880047878307,0) -- (12.648318654781852,0);\foreach \x in {-2,-1,1,2,3,4,5,6,7,8,9,10,11,12}\draw[shift={(\x,0)},color=black] (0pt,2pt) -- (0pt,-2pt) node[below] {\footnotesize $\x$};\draw[->,color=black] (0,-0.4594847187514599) -- (0,8.48412332221215);\foreach \y in {,1,2,3,4,5,6,7,8}\draw[shift={(0,\y)},color=black] (2pt,0pt) -- (-2pt,0pt) node[left] {\footnotesize $\y$};\draw[color=black] (0pt,-10pt) node[right] {\footnotesize $0$};\clip(-2.3423880047878307,-0.4594847187514599) rectangle (12.648318654781852,8.48412332221215);\fill[line width=0.8pt,color=rvwvcq,fill=rvwvcq,fill opacity=0.2] (4,3) -- (6,3) -- (6,5) -- (4,5) -- cycle;\fill[line width=0.4pt,color=ffqqqq,fill=ffqqqq,fill opacity=0.1] (2,2) -- (8,2) -- (8,5) -- (2,5) -- cycle;\draw [line width=0.8pt,color=rvwvcq] (4,3)-- (6,3);\draw [line width=0.8pt,color=rvwvcq] (6,3)-- (6,5);\draw [line width=0.8pt,color=rvwvcq] (6,5)-- (4,5);\draw [line width=0.8pt,color=rvwvcq] (4,5)-- (4,3);\draw [line width=0.4pt,color=ffqqqq] (2,2)-- (8,2);\draw [line width=0.4pt,color=ffqqqq] (8,2)-- (8,5);\draw [line width=0.4pt,color=ffqqqq] (8,5)-- (2,5);\draw [line width=0.4pt,color=ffqqqq] (2,5)-- (2,2);\begin{scriptsize}\draw [color=ffqqqq] (5,1) circle (2.5pt);\draw[color=ffqqqq] (5.35622914410982,1.2746097380546946) node {$(5, 1)$};\draw [color=ffqqqq] (1,3) circle (2.5pt);\draw[color=ffqqqq] (1.3544727053263874,3.2691359630991363) node {$(1, 3)$};\draw [color=ffqqqq] (2,6) circle (2.5pt);\draw[color=ffqqqq] (2.3580878121958833,6.267277295013074) node {$(2, 6)$};\draw [color=ffqqqq] (5,8) circle (2.5pt);\draw[color=ffqqqq] (5.35622914410982,8.274507508752066) node {$(5, 8)$};\draw [color=ffqqqq] (9,4) circle (2.5pt);\draw[color=ffqqqq] (9.357985582893251,4.272751069968632) node {$(9, 4)$};\draw [color=rvwvcq] (5,3)-- ++(-3pt,0 pt) -- ++(6pt,0 pt) ++(-3pt,-3pt) -- ++(0 pt,6pt);\draw[color=rvwvcq] (5.35622914410982,3.2945439404882375) node {$(5, 3)$};\draw [color=rvwvcq] (4,4)-- ++(-3pt,0 pt) -- ++(6pt,0 pt) ++(-3pt,-3pt) -- ++(0 pt,6pt);\draw[color=rvwvcq] (4.352614037240324,4.298159047357734) node {$(4, 4)$};\draw [color=rvwvcq] (6,5)-- ++(-3pt,0 pt) -- ++(6pt,0 pt) ++(-3pt,-3pt) -- ++(0 pt,6pt);\draw[color=rvwvcq] (6.359844250979315,5.30177415422723) node {$(6, 5)$};\draw[color=rvwvcq] (5.038629426746055,4.158415171717677) node {$S$};\draw[color=ffqqqq] (4.975109483273302,2.633936528371607) node {$G$};\end{scriptsize}\end{tikzpicture}
\item 
Да предположим, че вие трябва да предложите нов пример $<x, y>$ и да
запитате учителя за неговата класификация. Предложете заявката, която
гарантирано ще намали пространството на версиите независимо от това, как
учителят ще й класифицира. Предложете и друга заявка, която няма да
намали това пространство.
\paragraph{Решение}
Ако искаме да намалим пространството на версиите, тогава $h_g$ трябва да е съвместима с примера $d$, а $h_s$ трябва да е несъвместима с примера $d$ или обратното. И в двата случая това означава, че примерът $d$ трябва да е извън правоъгълника обособен от $S$ и да е вътре в правоъгълника, обособен от $G$. Така както и да бъде класифициран примера от учителя, той винага ще е несъвместим точно с една от хипотезите $h_s$ и $h_g$.
По този начин ако учителят каже, че примерът е положителен, тогава ще се промени $S$, а ако каже, че е отрицателен, ще се промени $G$. И в двата случая ще се намали пространството на версиите. \\
Това означава, че примерът $<3, 3>$ задължително ще намали пространството на версиите, а примерът $<8, 8>$ няма да го намали.
\item
А сега да предположим, че сте учител, опитващ да научи алгоритъм на едно определено понятие (например $3 \leq x \leq 5 , 2 \leq y \leq 9$ ). Какъв е най-малкият брой
на обучаващите примери трябва да предоставите на алгоритъма за
елиминиране на кандидати, за да може той абсолютно точно да научи това
понятие?
\paragraph{Решение}
За да се стигне до точно тази хипотеза, това означава, че $G$ и $S$ съвпадат и да се състоят точно от една хипотеза. Нека това е хипотезата:
\[a \leq x \leq b, c \leq y \leq d\]
Множеството $S$ се ограничава като около положителните примери се построява най-малкият правоъгълник обграждащ примерите. За да се постигне правоъгълника отговарящ на търсената хипотеза само 2 примера са необходими: $<a, c>$ и $<b, d>$. След като те бъдат обработени $S$ ще има вида:
\[S = \{a \leq x \leq b, c \leq y \leq d\}\]
Множеството $G$ се ограничава като всеки един от отрицателните примери отсича една част от интервала за $x$ или за $y$. Това означава, че за да се стигне от:
\[-\infty \leq x \leq \infty, -\infty \leq y \leq \infty\]
до
\[a \leq x \leq b, c \leq y \leq d\]
Трябва да получим четири отрицателни примера, които да ограничат $x$ и $y$ отгоре и отдолу. Това биха могли да бъдат
\[
	\begin{array}{lr}
		<a - 1, t>\\
		<b + 1, t>\\
		<t, c - 1>\\
		<t, d + 1>\\
	\end{array}
\]
където $t$ е произволно цяло число. По този начин $G$ ще придобие вида:
\[G = \{a \leq x \leq b, c \leq y \leq d\}\]
\end{enumerate}
\end{document}
